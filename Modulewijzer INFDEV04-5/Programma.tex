\section{Course program}
The course is made up of seven lectures and practicums. During the lectures both theoretical
concepts and applied examples will be covered. During the practicum the students will be
asked to practice independently (with the help of teachers when needed) the applied examples
seen in the lectures. The students can experiment in the practicums. All the work done in the
practicums can be seen as formative exercises in preparation of the exam.  
	\\
	\\
	\begin{tabular}{ | p{1.2cm} | p{10cm} | }
		\hline
	  	\textbf{Units} & \textbf{Topics} \\
	  	\hline 
  		1 & Introduction-to-distributed-applications
  			\begin{itemize}[nolistsep]
				\item The characteristics of distributed applications
				\item The Model-View-Controller (MVC) design pattern 
				\item Object-relational mapper (ORM)
				\item The M in MVC
			\end{itemize}
 		\\
  		\hline
  		2 & Modeling-queries-and-managing-data
  		\begin{itemize}[nolistsep]
					\item Impedence mismatch
					\item Mapping data from relational/physical model to domain models
					\item A genereric model to safely query relational models
					\item The costs of accessing data
					\item Improving queries safeness through typing (LINQ)
				\end{itemize}
  		\\
  		\hline
  		3 & Controlling-the-data-flow-in-the-application
  		\begin{itemize}[nolistsep]
					\item Taming the complexity of models in a distributed application
					\item The C in MVC to narrow the access to the model
					\item HTTP Protocol
					\item Architecturing distributed applications through the REST-model and testing
				\end{itemize}
  		 \\
  		\hline
  		4 &  Rendering-instances-of-the-model
  		\begin{itemize}[nolistsep]
					\item The V in the MVC
					\item Serving static pages with template engines
					\item The challenge of interacting with the model
				\end{itemize}
  		 \\
  		\hline
  		5 & Towards-a-new-rendering-architecture
  		\begin{itemize}[nolistsep]
					\item Client side/server independent programming
					\item Managing state on the client (Javascript and the DOM)
				\end{itemize}
  		 \\
  		\hline
  		6 & Single-page-application
  		\begin{itemize}[nolistsep]
				\item Putting in relation model and view in the client
				\item The concepts of containers and components in React
				\item The callback model
			\end{itemize}
			\\
  		\hline
  		7 & Increasing-safeness-on-client-side
  		\begin{itemize}[nolistsep]
				\item Validating state access through types
				\item Typescript as superset of Javascript to guarantee correct usages of the model
				\item Implication of using types in structuring the application
				\end{itemize} 
				\\
			\hline
	\end{tabular}
		
\begin{comment}
    A unit does not necessarily match with a lesson. Some subjects of a unit could be discussed in tow different lessons.
\end{comment}
\newpage
\subsection{Learning materials}\label{learning-materials}
\begin{itemize}[nolistsep]
\item
  Materials used in the lessons:
  \url{https://github.com/hogeschool/} then navigate to development 5 repository
\item
  Entity framework core online documentation:
  \url{https://docs.microsoft.com/en-us/ef/core/index}
\item
  Asp.net core online documentation:
  \url{https://docs.microsoft.com/en-us/aspnet/core/}
\item
  React online documentation:
  \url{https://facebook.github.io/react/docs/hello-world.html}
\item
  Typescript online documentation:
 \url{https://www.typescriptlang.org/docs/handbook/react-\&-webpack.html}
\end{itemize}

