\makeatother
 % Exact colors from NB
%\definecolor{incolor}{rgb}{0.0, 0.0, 0.5}
%\definecolor{outcolor}{rgb}{0.545, 0.0, 0.0}
% Prevent overflowing lines due to hard-to-break entities
\sloppy 
% Setup hyperref package
\hypersetup{
breaklinks=true,  % so long urls are correctly broken across lines
colorlinks=true,
urlcolor=urlcolor,
linkcolor=linkcolor,
citecolor=citecolor,
}
% Slightly bigger margins than the latex defaults
%\geometry{verbose,tmargin=1in,bmargin=1in,lmargin=1in,rmargin=1in}
%\begin{document}
%\maketitle
\section{Development 5 (INFDEV04-5)}\label{development-5-infdev04-5}

\subsection{Study points and contact
time}\label{study-points-and-contact-time}

The course awards students 4 ECTS, in correspondence with 112 hours of
student work.

The course consists of seven frontal lectures and seven practicums. The
rest is self study. 

\subsection{Introduction}\label{introduction}

This is the course descriptor for the \emph{Development 5} course.
\\
Development 5 covers a presentation of modern, distributed (over HTTP)
applications which follow the MVC architectural pattern. Given the huge
breadth of usable technologies in the field, we will focus the
implementation on a single stack: ASP.Net Core, Postgresql, and React
over TypeScript.

An important reminder: the course is not meant to be a \emph{build a
website in 16 hours} workshop. This would collide with the philosophy of
the Informatica degree, which aims at giving the foundational tools to
empower students as ongoing learners. For this reason, we will not dive
deeply into the intricacies of a given technology, but only use the
minimum needed to understand the underlying concept(s) and move on. As a
student, it is highly desirable to realize that most likely each of you
will use \emph{different technologies} on the workplace, and will have
to \emph{keep learn new ones}, so focusing on a single language, stack,
or library would do more harm than good.

    \subsubsection{Learning goals}\label{learning-goals}

The course has the following learning goals: - (PR\_M) students can
implement the model of a given MVC application in order to interact with
a permanent storage structure; - (PR\_C) students can work with a given
controller of an MVC application in order to handle interactions with
the model; - (PR\_V) students can work with the view of a given MVC
application in order to add interaction elements and improve safeness.

The course, and therefore also the learning goals, are limited to
idiomatic C\# and ASP.Net Core constructs, and idiomatic TypeScript and
React constructs (augmented with a few essential \emph{npm} libraries
such as Immutablejs and Fetch).

\subsubsection{Competences}\label{competences}
  Realization


    \subsubsection{Exam}\label{exam}

The exam consists of a series of exercises where students, given partial
code and the desired state transitions, are requested to fill in the
missing code that matches the given state transitions.

The exercises are split as follows: 
\begin{itemize}
\item Part 1, 25\% of the exercises, requires students to complete a model implementation; 
\item Part 2, 25\% of the exercises, requires students to complete a controller implementation; 
\item Part 3, 50\% of the exercises, requires students to complete a view implementation.
\end{itemize}

For example the exam could be made of 8 exercises: 2 about the model, 2
about the controller and 4 about the view.

This reflects the relatively higher emphasis on interaction that modern
distributed applications are showing. This leads development time to be
split non-uniformly across the architectural elements.

\paragraph{Scoring}\label{scoring}

The exam results in a full grade (from 0 to 10). Each exercise awards
one single point, if correctly completed. The grade is computed as the
percentage of points obtained, divided by 10. Students who score at
least 55\% of the total points will get a passing grade. For example, if
a student completes 5 exercises out of 8, this means obtaining 5 points
out of 8, which means 62,5\% and corresponds thus to a 6,25 (62,5/10).

{[}3:53{]} because it's impossible to have an exam made by 5 exercises
(given the 25/25/50 percentages) \#\#\#\# Retake If the exam is not
passed, then it will need to be retaken during the current schoolyear.
The retake will be scheduled at the end of the following period.
\paragraph{Exam matrix}\label{exam-matrix}

The exam covers all learning goals.

\begin{longtable}[]{@{}lccc@{}}
\toprule
Exam part & Part 1 & Part 2 & Part 3\tabularnewline
\midrule
\endhead
PR\_M & V & &\tabularnewline
PR\_C & & V &\tabularnewline
PR\_V & & & V\tabularnewline
\bottomrule
\end{longtable}

    \subsection{Lecture plan}\label{lecture-plan}

The course is made up of seven lectures and practicums. During the
lectures both theoretical concepts and applied examples will be covered.
During the practicum the students will be asked to practice
independently (with the help of teachers when needed) the applied
examples seen in the lectures. The students can experiment in the
practicums. All the work done in the practicums can be seen as formative
exercises in preparation of the exam.

\subsubsection{Chapter 1 - Introduction to distributed
applications}\label{chapter-1---introduction-to-distributed-applications}

\begin{itemize}
\tightlist
\item
  The characteristics of distributed applications
\item
  The Model-View-Controller (MVC) design pattern
\item
  Object-relational mapper (ORM)
\item
  The M in MVC
\end{itemize}

\subsubsection{Chapter 2 - Modeling queries and managing
data}\label{chapter-2---modeling-queries-and-managing-data}

\begin{itemize}
\tightlist
\item
  Impedence mismatch
\item
  Mapping data from relational/physical model to domain models
\item
  A genereric model to safely query relational models
\item
  The costs of accessing data
\item
  Improving queries safeness through typing (LINQ)
\end{itemize}

\subsubsection{Chapter 3 - Controlling the data-flow in the
application}\label{chapter-3---controlling-the-data-flow-in-the-application}

\begin{itemize}
\tightlist
\item
  Taming the complexity of models in a distributed application
\item
  The C in MVC to narrow the access to the model
\item
  HTTP Protocol
\item
  Architecturing distributed applications through the REST-model and
  testing
\end{itemize}

\subsubsection{Chapter 4 - Rendering instances of the
model}\label{chapter-4---rendering-instances-of-the-model}

\begin{itemize}
\tightlist
\item
  The V in the MVC
\item
  Serving static pages with template engines
\item
  The challenge of interacting with the model
\end{itemize}

\subsubsection{Chapter 5 - Towards a new rendering
architecture}\label{chapter-5---towards-a-new-rendering-architecture}

\begin{itemize}
\tightlist
\item
  Client side/server independent programming
\item
  Managing state on the client (Javascript and the DOM)
\end{itemize}

\subsubsection{Chapter 6 - Single page
application}\label{chapter-6---single-page-application}

\begin{itemize}
\tightlist
\item
  Putting in relation model and view in the client
\item
  The concepts of containers and components in React
\item
  The callback model
\end{itemize}

\subsubsection{Chapter 7 - Increasing safeness on client
side}\label{chapter-7---increasing-safeness-on-client-side}

\begin{itemize}
\tightlist
\item
  Validating state access through types
\item
  Typescript as superset of Javascript to guarantee correct usages of
  the model
\item
  Implication of using types in structuring the application
\end{itemize}

    \subsection{Learning materials}\label{learning-materials}

\begin{itemize}
\tightlist
\item
  Materials used in the lessons:
  \url{https://github.com/hogeschool/Development-5/tree/2017-2018/Lectures}
\item
  Entity framework core online documentation:
  \url{https://docs.microsoft.com/en-us/ef/core/index}
\item
  Asp.net core online documentation:
  \url{https://docs.microsoft.com/en-us/aspnet/core/}
\item
  React online documentation:
  \url{https://facebook.github.io/react/docs/hello-world.html}
\item
  Typescript online documentation:
 \url{https://www.typescriptlang.org/docs/handbook/react-\&-webpack.html}
\end{itemize}


% Add a bibliography block to the postdoc
% \end{document}
