\section{Assessment}

\subsection*{Theory}
The theory check consists of 10 of multiple choice questions. Each multiple choice question gives one point if correctly answered. 
 
\subsection*{Practicum}
The practicum check consists of a series of exercises where students, given partial
code and the desired state transitions, are requested to fill in the
missing code that matches the given state transitions.
\\
The exercises are split as follows: 
\begin{itemize}
\item Part 1, 25\% of the exercises, requires students to complete a model implementation; 
\item Part 2, 25\% of the exercises, requires students to complete a controller implementation; 
\item Part 3, 50\% of the exercises, requires students to complete a view implementation.
\end{itemize}
Each exercise is associated to a certain amount of points. The total amount of points is 10.
For example the exam could be made of 8 exercises: 2 about the model, 2
about the controller and 4 about the view.

This reflects the relatively higher emphasis on interaction that modern
distributed applications are showing. This leads development time to be
split non-uniformly across the architectural elements.

\subsection*{Scoring}

The exam results of two grades (each from 0 to 10). You need to pass both to get the credit points of the course.
For the written exam you need to score at least 5,5 to pass. 
For the practicum exam you need to score at least 5,5 to pass.  



\subsection*{Matrix}
The exam covers all learning goals.
\begin{center}
\begin{tabular}{ |c|c|c|c|c| } 
\hline 
Exam part & Written Part & Practicum Part 1 & Practicum Part 2  & Practicum Part 3 \\
\hline
U\_M  & V & &&\\
U\_C  & V& &&\\
U\_V  & V &&\\
PR\_M  && V &&\\
PR\_C  & &&V& \\
PR\_V & && & V \\
\hline
\end{tabular}
\end{center}

\subsection{Retake}
 If the exam is not passed, then it will need to be retaken during the current schoolyear.
The retake will be scheduled at the end of the following period.



	
